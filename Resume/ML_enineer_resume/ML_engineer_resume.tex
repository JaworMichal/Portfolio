%-------------------------
% Resume in LaTeX
% Author: Sourabh Bajaj
% License: MIT
%------------------------

% Examples of other good resumes for reference:
% 1. From Sergey Nemchinski's videos:
% 1.1. https://youtu.be/-ghtEmXux5U?t=3343
% 1.2. https://youtu.be/6ZmJ6pAi6Ds?t=6660
% 1.3. https://youtu.be/k8knBXQkK0g?t=5737 (minimal details)
% 1.4. https://youtu.be/jhnsjSCgtu0?t=2341
% 1.5. https://youtu.be/YszpkGriUgQ?t=819
% 2. The TechLead: https://youtu.be/xpaz7nrNmXA?t=587
% 3. Clement Mihailescu: https://youtu.be/aKjsy-b00QM?t=806
% 4. Mary Feofanova: https://github.com/mary3000/resume
% 5. Good Samsung Resume (simple, elegant, and informative -- areas of work without details): https://wcshim40.github.io/cv/

% Example feedback to others' resumes:
% 1. Vika Borodina FAANGME: https://youtu.be/AcOJpIqbsm0?t=549

\documentclass[letterpaper,11pt]{article}
% Global notes:
% 1. Font
% 1.a. Font size: I saw an even smaller font in other resumes, so think about making it smaller
% 1b. Italic font style:
% 1b.1. The TechLead suggests not using italic, and I think that a little italic is not that bad
% 1.b.2. Sergey Nemchinski said about one Italic example that it was good: https://youtu.be/YszpkGriUgQ?t=1239
% 1c. Bold font style: too much bold is not ok. In this resume, now it's not too much, but know that
% 2. Dates are shifted to the right, and it could be inconvenient to read them
% 3. Don't use colons and quotes (https://www.youtube.com/live/VscDaW81Y-Y?feature=share&t=2472), use different formatting
% 4. Spaces between sections:
% 4.1. A piece of advice from Veronika Kosiak in Nemchinki's video (https://www.youtube.com/live/VscDaW81Y-Y?feature=share&t=5436): all the sections should be separated from each other with visible space

%-------------------------
% Use packages
%-------------------------

% blackboard math symbols
\usepackage{amsfonts}
% professional-quality tables
\usepackage{booktabs}
\usepackage[usenames,dvipsnames]{color}
\usepackage{enumitem}
\usepackage{fancyhdr}
% add icons like LinkedIn or Email envelope, fontawesome, and fontawesome5 are conflicting, only one can be imported
%\usepackage{fontawesome}
\usepackage{fontawesome5}
% use 8-bit T1 fonts
\usepackage[T1]{fontenc}
\usepackage[empty]{fullpage}
\usepackage[pdftex]{hyperref}
% allow utf-8 input
\usepackage[utf8]{inputenc}
\usepackage{latexsym}
% micro typography
\usepackage{microtype}
\usepackage{marvosym}
% compact symbols for 1/2, etc.
\usepackage{nicefrac}
\usepackage{tikz}
\usepackage{titlesec}
% simple URL typesetting
\usepackage{url}
\usepackage{verbatim}
\usepackage{xcolor}
% fine itemize lists with vertical lines
\usepackage{fdsymbol}
\usepackage{tikz}
\usetikzlibrary{calc}

%-------------------------
% Custom variables
%-------------------------

% One-line items are different from two-line ones because the skills are in the first line
\newcommand\additionalSpaceBetweenOneLineItems{1pt}
\newcommand\spaceAfterSubItems{6pt}
\newcommand\spaceBeforeGDPRStatement{-6pt}
\newcommand\spaceBetweenEducations{-6pt}
\newcommand\spaceBetweenLinesInEducation{-1pt}
\newcommand\spaceBetweenLinesInHeading{-12pt}
\newcommand\spaceBetweenOrdinarySubItems{-2pt}
\newcommand\spaceBetweenResumeItems{-2pt}
\newcommand\spaceBetweenResumeSebitems{-4pt}
\newcommand\spaceBetweenSections{-12pt}
\newcommand\spaceBetweenJobTitleAndDetails{2pt}

\newcommand\spaceBeforeSection{-6pt}
\newcommand\spaceBetweenItemAndSubItem{-4pt}
\newcommand\spaceBetweenSubItemAndItem{-10pt}
\newcommand\spaceBetweenSubItemAndSubSubItem{-1pt}
\newcommand\spaceBetweenSubSubItemAndSubItem{-1pt}

\newcommand\spaceAfterTikzPicture{-21pt}

\newcommand{\linkedlist}[1]{
  \begin{tikzpicture}[remember picture]%
    \node (#1) [gray, circle, fill, inner sep=2.0pt]{};
  \end{tikzpicture}
}

\renewcommand{\labelitemiii}{\(\circ\)}

%-------------------------
% Custom commands
%-------------------------
% Notes:
% 1. There are the resumeItem and the resumeSubItem commands defined, but the items used in this resume are created using conventional itemize tools

% Resume items
\newcommand{\resumeItem}[2]{
    \item\small{
        \textbf{#1}{: #2 \vspace{\spaceBetweenResumeItems}}
    }
}

% Text style for dates (e.g., textit, textbf, textnormal)
\newcommand{\textDate}{\textit}

\newcommand{\resumeSubheading}[4]{
    \vspace{-1pt}\item
    \begin{tabular*}{0.97\textwidth}{l@{\extracolsep{\fill}}r}
        \textbf{#1} & #2 \\
        \textit{\small#3} & \textit{\small #4} \\
    \end{tabular*}\vspace{-5pt}
}

\newcommand\blfootnote[1]{ % footnote without numbers
    \begingroup
    \renewcommand\thefootnote{}\footnote{#1}%
    \addtocounter{footnote}{-1}%
    \endgroup
}

\newcommand{\resumeSubItem}[2]{
    \resumeItem{#1}{#2}\vspace{\spaceBetweenResumeSebitems}
}

\renewcommand{\labelitemii}{$\circ$}

\newcommand{\resumeSubHeadingListStart}{\begin{itemize}[leftmargin=*]}
\newcommand{\resumeSubHeadingListEnd}{\end{itemize}}
\newcommand{\resumeItemListStart}{\begin{itemize}}
\newcommand{\resumeItemListEnd}{\end{itemize}\vspace{-5pt}}
\newcommand{\RomanNumeralCaps}[1]{\MakeUppercase{\romannumeral #1}}

% Create special formatting skills
\tikzset{rndblock/.style={rounded corners,rectangle,draw,outer sep=0pt}}
\newcommand{\tframed}[2][]{\tikz[baseline=(h.base)]\node[rndblock,#1] (h) {\color{black}{#2}};}
\newcommand*{\mystrut}{\rule[-0.2\baselineskip]{0pt}{0.8\baselineskip}}
\newcommand{\skill}[1]{\tframed[lightgray]{\mystrut#1}}

\makeatletter
\newcommand*\bigcdot{\mathpalette\bigcdot@{.5}}
\newcommand*\bigcdot@[2]{\mathbin{\vcenter{\hbox{\scalebox{#2}{$\m@th#1\bullet$}}}}}
\makeatother

%-------------------------
% Other settings
%-------------------------

\definecolor{linkcolor}{HTML}{0000FF} % colour of links
\definecolor{urlcolor}{HTML}{0000FF} % colour of hyperlinks
 
\hypersetup{
    pdfstartview=FitH,
    linkcolor=linkcolor,
    urlcolor=urlcolor,
    colorlinks=true
}

\pagestyle{fancy}
\fancyhf{} % clear all header and footer fields
\fancyfoot{}
\renewcommand{\headrulewidth}{0pt}
\renewcommand{\footrulewidth}{0pt}
\setlength{\footskip}{4.08003pt} % just to get rid of a warning

% Adjust margins
\addtolength{\oddsidemargin}{-0.375in}
\addtolength{\evensidemargin}{-0.375in}
\addtolength{\textwidth}{1in}
\addtolength{\topmargin}{-.5in}
\addtolength{\textheight}{1.0in}

\urlstyle{same}

\raggedbottom
\raggedright
\setlength{\tabcolsep}{0in}

% Sections formatting
\titleformat{\section}{
    \vspace{\spaceBetweenSections}
    \scshape\raggedright\large
}{}{0em}{}[\color{black}\titlerule \vspace{-5pt}]


%-------------------------------------------
%%%%%%  RESUME STARTS HERE  %%%%%%%%%%%%%%%%%%%%

\begin{document}


%----------HEADING-BEGIN----------

\begin{tabular*}{\textwidth}{l @{\extracolsep{\fill}} c @{\extracolsep{\fill}} r}
    \faLinkedinIn \enspace LinkedIn: \href{https://www.linkedin.com/in/jawormichal/}{ jawormichal}
    &
    \textbf{\LARGE Michał Jaworski} \hspace{50pt}
    &
    \faMapMarker* Warsaw, Poland \\
\end{tabular*}
\vspace{\spaceBetweenLinesInHeading}

\begin{tabular*}{\textwidth}{l @{\extracolsep{\fill}} c @{\extracolsep{\fill}} r}
    \faEnvelope[regular] jawormichal128@gmail.com
    &&
    % Note: placeholder for a phone number
    %\faTelegram \enspace Telegram:
    % Note: placeholder for a telegram link
    %\faPhone \enspace
    \\
\end{tabular*}

%----------HEADING-END----------


%----------SKILLS-BEGIN----------
% Notes:
% 1. Think about moving it above, even higher than the Experience section
% 1.1 I remember a recommendation from Google recruiters to put programming languages at the top and from Sergey Nemchinski to put all the skills at the top, followed by Experience, and only then by Education
% 1.2. At the same time, Clement Mihailescu also put the Skills at the bottom (although he was experienced by that time), source: "The Resume That Got Me Into Google (software engineer resume tips)" (https://youtu.be/aKjsy-b00QM?t=806)
% 1.3. At the same time, the TechLead didn't summarize his skills at all in his resume, source: "7 Tips for the Coding Resume (for Software Engineers)" (https://youtu.be/xpaz7nrNmXA?t=587)
% 2. Think about removing some irrelevant skills (or at least moving them to the ends of respective lines), see more in the notes to the Experience section
% 2.1. Idea: a person who knows everything doesn't know anything, took it from Sergey Nemchinski's video (https://youtu.be/mi0mRDIL_z8?t=7816)
% 3. Think about adding soft skills here: Scrum, public speaking
% 3.1. Public speaking, for example, was proposed by Sergey Nemchinski (I don't remember the source), but he also said together with Alisa Yeremenko that one shouldn't list soft skills in a resume cos this is something recruiters find during an interview (source: https://youtu.be/-ghtEmXux5U?t=1118)
% 4. Maybe the "Technologies/Libraries" item can be split or merged with "Tools", as now the sub-items in those two items are similar

\vspace{\spaceBeforeSection}
\vspace{12pt} % a necessary space before the first section, for some reason the first section is an exception, and it's not enough to just set a global space before a sectionŚ

\section{Skills}
\resumeSubHeadingListStart

% Note: the only reason why sub-items are used instead of items is that they provide good spacing out of the box
% A suggestion from Nemchinski (https://www.youtube.com/live/VscDaW81Y-Y?feature=share&t=3017) to remove names of sub-items, like it's clear anyway that C++ is a language, and so on
% A suggestion mentioned in Nemchinski's video (https://www.youtube.com/live/VscDaW81Y-Y?feature=share&t=1219) by Veronica Kosiak is to put the skills on top of the resume to simplify the job of a recruiter, cos they first look at the skills to match them with the position description
\resumeSubItem{Languages}{C\textbf{\footnotesize++}, C, Python, C\#}
\resumeSubItem{Deep Learning frameworks}{PyTorch, TensorFlow, Keras}
\resumeSubItem{Libraries/Technologies}{NumPy/SciPy, Sklearn, Pandas, OpenCV, Unity, C\textbf{\footnotesize++} Boost, MapReduce
}
\resumeSubItem{Tools}{Git, Docker, GDB, Valgrind, \LaTeX, CMake, Deployment Pipelines (CI/CD), UNIX/Linux, Windows Server
}

% Usually, sections consist of resume items, but this one consists of resume sub-items
% Spaces between sub-items are smaller, so to compensate for this loss of space, it's added manually
\vspace{\spaceAfterSubItems}

\resumeSubHeadingListEnd

%----------SKILLS-END----------


%----------EXPERIENCE-BEGIN----------
% Notes:
% 1. Still not sure whether I should write what I was working on
% 1a. Put more numbers and clear measurable achievements in this section (Alexandra Chudinova says it's a must for the US and Canada: https://youtu.be/7aZSO8v58rE?t=3699, https://youtu.be/7aZSO8v58rE?t=7047)
% 1b. Sometimes people write about what they were developing (projects)
% 1c. Writing about your duties (e.g., writing integration tests) is excessive because it's obvious from the job title
% 2. Think about mentioning irrelevant skills or writing too much about an irrelevant job position that you had could make it more difficult for a recruiter to decide whether to take you
% 3. Put the total years of experience near the Experience tab to quickly attract the recruiter's attention
% 4. Think about putting the locations of places where you worked

% Sources:
% 2.1. The TechLead's video called "Ex-Googler Resume Tips for Software Engineers": (https://youtu.be/rCOgVQ8a1zs?t=137)
% 2.2. Sergey Nemchinski's video called Resume analysis (https://youtu.be/J8k5TCa3UYA?t=645)
% 2.3. Various people say that it's bad to have one resume for all roles, suggesting that if you have multiple resumes, you can remove irrelevant stuff from them
% 3. Looks like it's ok to have gaps between different work positions (from a video of Sergey Nemchinski and Alexandra Chudinova: https://youtu.be/7aZSO8v58rE?t=6001)
% 4. I believe that a short description of the projects would help to make a picture of what the person was doing (proofs: Nemchinski's video https://youtu.be/jhnsjSCgtu0?t=1309, and I tried to read a resume sent by Yashasvi Singh, and I understood that it's easier to read when you understand what the person was doing)
% 5. A Google recruiter asked me about my responsibilities and my current job. It looks like they might be different in different companies
% 6. If there were many projects in a particular position, it can be written like that "many various projects in X area" (the area should be mentioned), source: Nemchinski's video (https://youtu.be/jhnsjSCgtu0?t=4346)

\vspace{\spaceBeforeSection}

\section{Experience}
\resumeSubHeadingListStart

% AI Engineer at Samsung R&D Center
\item{
    \textbf{AI Engineer at \href{https://research.samsung.com/srpol}{\color{blue} Samsung Research}}
    \hfill
    \textDate{Sep 2025 -- present $\bigcdot$ 1 mo} \\
    \vspace{\spaceBetweenJobTitleAndDetails}
    \textbf{Language Intelligence team}
    \hfill
    \textDate{Warsaw, Poland} \\
}
\vspace{\spaceBetweenItemAndSubItem}
\begin{itemize}
    \item[\linkedlist{a}] {
        \textbf{ToDo: add the name of the project and skills} \skill{Python}
        \vspace{\spaceBetweenSubItemAndSubSubItem}
        \begin{itemize}
            \item ToDo: add achievements
        \end{itemize}
        \vspace{\spaceBetweenSubSubItemAndSubItem}
    }
    \vspace{\spaceBetweenSubItemAndItem}
\end{itemize}
\vspace{\spaceBetweenResumeItems}

% Software Engineer at Samsung R&D Center
\item{
    \textbf{Software Engineer at \href{https://research.samsung.com/srpol}{\color{blue} Samsung Research}}
    \hfill
    \textDate{Feb 2023 -- Sep 2025 $\bigcdot$ 2yr 7 mo} \\
    \vspace{\spaceBetweenJobTitleAndDetails}
    \textbf{Visual Display team}
    \hfill
    \textDate{Warsaw, Poland} \\
}
\vspace{\spaceBetweenItemAndSubItem}
\begin{itemize}
    \item[\linkedlist{a}] {
        \href{https://news.samsung.com/global/samsung-electronics-unveils-groundbreaking-glasses-free-odyssey-3d-gaming-monitor-at-gamescom-2024}{\textbf{\color{blue}Glasses-free 3D display}} \skill{Python} \skill{OpenCV} \skill{Unity} \skill{C++} \skill{CMake}
        \vspace{\spaceBetweenSubItemAndSubSubItem}
        \begin{itemize}
            \item Developed a \textbf{mathematical model} related to human perception of stereoscopy
            \item \textbf{Experimental verification} of the model using methods from \\ \textbf{computer vision} and \textbf{computational photography}
            \item \textbf{Progress in the research} led to \textbf{start of a new project}
            \item Developed a proof of concept application in Unity using the model
            \item \textbf{Leading development} in a team of two people, planning tasks
        \end{itemize}
        \vspace{\spaceBetweenSubSubItemAndSubItem}
    }
    \vspace{\spaceBetweenOrdinarySubItems}
    \item[\linkedlist{b}] {
        \href{https://www.samsung.com/latin_en/support/tv-audio-video/how-to-use-the-sign-language-guide-feature-on-samsung-smart-tv/}{\textbf{\color{blue} Sign Language Avatar}}: an animated helper for deaf people \skill{C\textbf{\footnotesize++}} \skill{GDB} \skill{Valgrind} \skill{Tizen}
        \vspace{\spaceBetweenSubItemAndSubSubItem}
        \begin{itemize}
            \item \textbf{Leading development} of the project, \textbf{communicating} needs and priorities with management
            \item Working on \textbf{simplification} of the project architecture
            \item Writing full project documentation for a \textbf{successful ownership transfer}
        \end{itemize}
        \vspace{\spaceBetweenSubSubItemAndSubItem}
    }
\end{itemize}

\begin{tikzpicture}[remember picture,overlay]
    \draw[gray] ($(a)!0.1!(b)$) -- ($(a)!0.9!(b)$);
\end{tikzpicture}

\vspace{\spaceAfterTikzPicture}
\vspace{\spaceBetweenSubItemAndItem}

% Software Development Engineer at Amazon
\item{
    \textbf{Software Development Engineer at \href{https://www.aboutamazon.com/}{\color{blue} Amazon}}
    \hfill
    \textDate{Aug 2021 -- Aug 2022 $\bigcdot$ 1yr} \\
    \vspace{\spaceBetweenJobTitleAndDetails}
    \textbf{Alexa TextToSpeech}
    \skill{C\textbf{\footnotesize++}} \skill{C} \skill{Python} \skill{Deployment Pipelines (CI/CD)}
    \hfill
    \textDate{Gdańsk, Poland}
    \\
}
\vspace{\spaceBetweenItemAndSubItem}
\begin{itemize}
    \item{
        Working on \textbf{Language Models} for Speech Synthesis
    }
    \vspace{\spaceBetweenOrdinarySubItems}
    \item{
        \textbf{Reduced latency} of a homograph disambiguation model by \textbf{56\%}
    }
\end{itemize}

\resumeSubHeadingListEnd

%----------EXPERIENCE-END----------


%----------PROJECTS-BEGIN----------
% Notes:
% 1. Think about removing this section
% 1.1 Sergey Nemchinski, in his video (https://youtu.be/2zps16PJA1k?t=707), had doubts about adding projects if someone already has 1 year of experience. Like it's not necessary, but it also adds, so maybe not a problem. Most likely, now it can be left, but you can remove this section when adding the next job

\vspace{\spaceBeforeSection}

\section{Projects}
\resumeSubHeadingListStart

% BigARTM
\vspace{-2pt} % not sure if necessary, it just looks ok with it
\item{
    \textbf{\href{https://github.com/bigartm/bigartm}{\color{blue} BigARTM}}
    \skill{C\textbf{\footnotesize++}} \skill{Boost} \skill{Protobuf} \skill{Travis CI} \skill{AppVeyor}
    \hfill
    \textDate{Jan 2017 -- Jun 2018} \\
    \vspace{2pt} % added to have normal space between the skills and the next line
    \textbf{Open Source library for topic modeling} \\
    Developed a tool for parallel calculation of pairwise word statistics (\textbf{\href{https://github.com/JaworMichal/bigartm/blob/master/src/artm/core/cooccurrence_collector.cc}{code sample}, \href{https://bigartm.readthedocs.io/en/stable/tutorials/bigartm_cli.html}{documentation}})
}

\resumeSubHeadingListEnd

%----------PROJECTS-END----------


%----------EDUCATION-BEGIN----------
% Notes:
% 1. Think about moving it below (that's a usual recommendation for experienced engineers, although I'm not sure it applies to ML engineers)
% 2. Think about removing the bachelor's degree, as you're a master's, although these are different universities

\vspace{\spaceBeforeSection}

\section{Education}{}
\resumeSubHeadingListStart

% Master of Science
\item{
    \textbf{Master of Science} in \textbf{Applied Mathematics and Computer Science} at
    \hfill
    \textDate{Sep 2019 -- Jun 2021}
    \\
    \vspace{\spaceBetweenLinesInEducation}
    \textbf{{\color{blue} Higher School of Economics} :}
	\textbf{\color{blue} Faculty of Computer Science}
    \\
    \vspace{\spaceBetweenLinesInEducation}
    % The British version is Honours
    \textbf{Diploma with Honors, GPA 3.90 / 4.0}
}

% Bachelor of Science
% \vspace{\spaceBetweenEducations}
% \item{
%     Bachelor of Science in Applied Mathematics and Computer Science, GPA 3.89 / 4.0 % (4.89 / 5.0)
%     \hfill
%     \textDate{Sep 2015 -- Jun 2019}
%     \\
%     \vspace{\spaceBetweenLinesInEducation}
%     \textbf{\href{https://www.topuniversities.com/universities/lomonosov-moscow-state-university}{\color{blue} Lomonosov Moscow State University}}
%     \\
%     \vspace{\spaceBetweenLinesInEducation}
%     \href{https://www.msu.ru/en/info/struct/depts/vmc.html}{\color{blue} Faculty of Computational Mathematics and Cybernetics}
% }
\resumeSubHeadingListEnd

%----------EDUCATION-END----------


%----------GDPR-BEGIN----------
% Consent to personal data processing following the GDPR
% Notes:
% 1. Most likely not needed outside of the EU (not sure about the UK)
% 1. Possibly, there could be a smaller statement; you can try to find it

% \vspace{\spaceBeforeGDPRStatement} % Remove some space, as by default the shift before this statement is too big
% \blfootnote{\fontsize{8pt}{8pt}\selectfont
    % I hereby give consent for my personal data included in the application to be processed for the purposes of the recruitment process in accordance with Art. 6 paragraph 1 letter a of the Regulation of the European Parliament and of the Council (EU) 2016/679 of 27 April 2016 on the protection of natural persons with regard to the processing of personal data and on the free movement of such data, and repealing Directive 95/46/EC (General Data Protection Regulation).
% }

%----------GDPR-END----------


\end{document}
